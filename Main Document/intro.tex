% Introduction

\chapter{Introduction} \label{sc:intro}

\paragraph{}
Recent developments in machine learning have made significant 
contributions to a wide range of fields that are not 
traditionally considered data science.
% Notably the Netflix competition have attracted a collective
% effort in developing models that greatly improved prediction
% of movie ratings by different users,
% creating the best movie recommendation system at the time \cite{FeHeKh12}.
% Similar to the Netflix rating data,
% student grades in different courses follow the same structure,
% allowing the application of the same machine learning techniques.
In this research course, we intend to explore several of the 
machine learning techniques in applications to education.

\paragraph{}
Specifically, this research project aims to apply machine learning 
to analyze the student grade dataset from \cite{BaRoYo14},
which contains complete transcripts of undergraduate students
from a major Canadian University.
Similar to predicting user ratings,
we are able to predict the grades for courses.
From the predictions, this project intends to analyze
the effect of choosing easier courses on student grades,
specifically by comparing the predicted grades of courses
students did not take against the courses taken 
within the same program.
By analyzing the variation in course difficulty,
these results could potentially improve curriculum design 
for educational institutions and admission procedure for
graduate programs.

\paragraph{}
The project will focus on implementing three main methods of inference:
\vspace{-0em}
\begin{enumerate}
\setlength\itemsep{-0.2em}
  \item Matrix factorization (MF) \cite{FeHeKh12} and if time permits
  	probabilistic matrix factorization (PMF) \cite{MnSa07, SaMn08}
  \item Restricted Boltzmann machines (RBM) \cite{Sa09}
  \item Denoising auto-encoders (DAE) \cite{VLLBM10} and if time permits
  	variational auto-encoders (VAE) \cite{KiWe13}
\end{enumerate}


